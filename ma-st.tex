\documentclass[journal,12pt,twocolumn]{IEEEtran}
%
\usepackage{setspace}
\usepackage{gensymb}
%\doublespacing
\singlespacing

%\usepackage{graphicx}
%\usepackage{amssymb}
%\usepackage{relsize}
\usepackage[cmex10]{amsmath}
\usepackage{siunitx}
%\usepackage{amsthm}
%\interdisplaylinepenalty=2500
%\savesymbol{iint}
%\usepackage{txfonts}
%\restoresymbol{TXF}{iint}
%\usepackage{wasysym}
\usepackage{amsthm}
\usepackage{iithtlc}
\usepackage{mathrsfs}
\usepackage{txfonts}
\usepackage{stfloats}
\usepackage{steinmetz}
%\usepackage{bm}
\usepackage{cite}
\usepackage{cases}
\usepackage{subfig}
%\usepackage{xtab}
\usepackage{longtable}
\usepackage{multirow}
%\usepackage{algorithm}
%\usepackage{algpseudocode}
\usepackage{enumitem}
\usepackage{mathtools}
\usepackage{tikz}
\usepackage{circuitikz}
\usepackage{pgfplots}
\usepackage{verbatim}
\usepackage{tfrupee}
\usepackage[breaklinks=true]{hyperref}
%\usepackage{stmaryrd}
\usepackage{tkz-euclide} % loads  TikZ and tkz-base
%\usetkzobj{all}
\usetikzlibrary{calc,math}
\usetikzlibrary{fadings}
\usepackage{listings}
    \usepackage{color}                                            %%
    \usepackage{array}                                            %%
    \usepackage{longtable}                                        %%
    \usepackage{calc}                                             %%
    \usepackage{multirow}                                         %%
    \usepackage{hhline}                                           %%
    \usepackage{ifthen}                                           %%
  %optionally (for landscape tables embedded in another document): %%
    \usepackage{lscape}     
\usepackage{multicol}
\usepackage{chngcntr}
%\usepackage{enumerate}

%\usepackage{wasysym}
%\newcounter{MYtempeqncnt}
\DeclareMathOperator*{\Res}{Res}
%\renewcommand{\baselinestretch}{2}
\renewcommand\thesection{\arabic{section}}
\renewcommand\thesubsection{\thesection.\arabic{subsection}}
\renewcommand\thesubsubsection{\thesubsection.\arabic{subsubsection}}

\renewcommand\thesectiondis{\arabic{section}}
\renewcommand\thesubsectiondis{\thesectiondis.\arabic{subsection}}
\renewcommand\thesubsubsectiondis{\thesubsectiondis.\arabic{subsubsection}}

% correct bad hyphenation here
\hyphenation{op-tical net-works semi-conduc-tor}
\def\inputGnumericTable{}                                 %%

\lstset{
%language=C,
frame=single, 
breaklines=true,
columns=fullflexible
}
%\lstset{
%language=tex,
%frame=single, 
%breaklines=true
%}

\begin{document}
%


\newtheorem{theorem}{Theorem}[section]
\newtheorem{problem}{Problem}
\newtheorem{proposition}{Proposition}[section]
\newtheorem{lemma}{Lemma}[section]
\newtheorem{corollary}[theorem]{Corollary}
\newtheorem{example}{Example}[section]
\newtheorem{definition}[problem]{Definition}
%\newtheorem{thm}{Theorem}[section] 
%\newtheorem{defn}[thm]{Definition}
%\newtheorem{algorithm}{Algorithm}[section]
%\newtheorem{cor}{Corollary}
\newcommand{\BEQA}{\begin{eqnarray}}
\newcommand{\EEQA}{\end{eqnarray}}
\newcommand{\define}{\stackrel{\triangle}{=}}

\bibliographystyle{IEEEtran}
%\bibliographystyle{ieeetr}


\providecommand{\mbf}{\mathbf}
\providecommand{\pr}[1]{\ensuremath{\Pr\left(#1\right)}}
\providecommand{\qfunc}[1]{\ensuremath{Q\left(#1\right)}}
\providecommand{\sbrak}[1]{\ensuremath{{}\left[#1\right]}}
\providecommand{\lsbrak}[1]{\ensuremath{{}\left[#1\right.}}
\providecommand{\rsbrak}[1]{\ensuremath{{}\left.#1\right]}}
\providecommand{\brak}[1]{\ensuremath{\left(#1\right)}}
\providecommand{\lbrak}[1]{\ensuremath{\left(#1\right.}}
\providecommand{\rbrak}[1]{\ensuremath{\left.#1\right)}}
\providecommand{\cbrak}[1]{\ensuremath{\left\{#1\right\}}}
\providecommand{\lcbrak}[1]{\ensuremath{\left\{#1\right.}}
\providecommand{\rcbrak}[1]{\ensuremath{\left.#1\right\}}}
\theoremstyle{remark}
\newtheorem{rem}{Remark}
\newcommand{\sgn}{\mathop{\mathrm{sgn}}}
\providecommand{\abs}[1]{\left\vert#1\right\vert}
\providecommand{\res}[1]{\Res\displaylimits_{#1}} 
\providecommand{\norm}[1]{\left\lVert#1\right\rVert}
%\providecommand{\norm}[1]{\lVert#1\rVert}
\providecommand{\mtx}[1]{\mathbf{#1}}
\providecommand{\mean}[1]{E\left[ #1 \right]}
\providecommand{\fourier}{\overset{\mathcal{F}}{ \rightleftharpoons}}
%\providecommand{\hilbert}{\overset{\mathcal{H}}{ \rightleftharpoons}}
\providecommand{\ztrans}{\overset{\mathcal{Z}}{ \rightleftharpoons}}
\providecommand{\system}{\overset{\mathcal{H}}{ \longleftrightarrow}}
	%\newcommand{\solution}[2]{\textbf{Solution:}{#1}}
\newcommand{\solution}{\noindent \textbf{Solution: }}
\newcommand{\cosec}{\,\text{cosec}\,}
\providecommand{\dec}[2]{\ensuremath{\overset{#1}{\underset{#2}{\gtrless}}}}
\newcommand{\myvec}[1]{\ensuremath{\begin{pmatrix}#1\end{pmatrix}}}
\newcommand{\mydet}[1]{\ensuremath{\begin{vmatrix}#1\end{vmatrix}}}
\providecommand{\gauss}[2]{\mathcal{N}\ensuremath{\left(#1,#2\right)}}
%\providecommand{\system}[1]{\overset{\mathcal{#1}}{ \longleftrightarrow}}
\newcommand*{\permcomb}[4][0mu]{{{}^{#3}\mkern#1#2_{#4}}}
\newcommand*{\perm}[1][-3mu]{\permcomb[#1]{P}}
\newcommand*{\comb}[1][-1mu]{\permcomb[#1]{C}}

\title{
%\logo{
GATE Problems in Probability
%}
}

\maketitle

\begin{abstract}
These problems have been selected from GATE question papers and can be used for conducting tutorials in courses related to a first course in probability.
\end{abstract}
%\centering \textbf{\Large Probability}\\

\begin{enumerate}
\setlength\itemsep{2em}

\item Let X be a random variable with the following cumulative distribution function:
\begin{align}
F\brak{x} 
= 
\begin{cases}
0           & x < 0 \\
x^2         & 0 \leq x < \frac{1}{2} \\
\frac{3}{4} & \frac{1}{2} \leq x < 1 \\
1           &  x \geq 1 
\end{cases}
\end{align}
Then $ P\brak{\frac{1}{4} < X < 1}$ is equal to

%
\solution
\input{solutions/ma/2016/30/Assignment4.tex}
\item Let X and Y be continuous random variables with the joint probability density function 
\begin{align}
f\brak{x,y}= 
\begin{cases}
ae^{-2y} & 0<x<y<\infty \\
0 & \text{otherwise}.
\end{cases}   
\end{align}
Then $E\brak{X|Y=2}$ is \dots
\solution
\input{solutions/ma/2010/49.tex}
%
\item A continuous random variable X has the probability density function
\begin{align*}
    f(x) &= \begin{cases} 
      \frac{3}{5}e^{-\frac{3}{5}x} & x > 0 \\
      0 & x\leq 0\\
   \end{cases} 
\end{align*} 
The probability density function of $Y=3X+2$  is
\begin{enumerate}
    \item \begin{align*}
         f(y) &= \begin{cases} 
      \frac{1}{5}e^{-\frac{1}{5} (y-2)} & y > 2 \\
      0 & y \leq 2\\
   \end{cases} 
    \end{align*}
\item \begin{align*}
    f(y) &= \begin{cases} 
      \frac{2}{5}e^{-\frac{2}{5} (y-2)} & y > 2 \\
      0 & y \leq 2\\
   \end{cases} 
\end{align*} 
\item  \begin{align*}
    f(y) &= \begin{cases} 
      \frac{3}{5}e^{-\frac{3}{5} (y-2)} & y > 2 \\
      0 & y \leq 2\\
   \end{cases} 
\end{align*} 
\item \begin{align*}
    f(y) &= \begin{cases} 
      \frac{4}{5}e^{-\frac{4}{5} (y-2)} & y > 2 \\
      0 & y \leq 2\\
   \end{cases} 
\end{align*} 
\end{enumerate}
%
\solution
\input{solutions/ma/2012/8.tex}
%
\item    Let the probability density function of a random variable X be
   $$
   f(x)=
   \begin{cases}
   x~~~~~~~~~~~~~~~0\leqslant x < \dfrac{1}{2}\\
   c(2x-1)^2~~~~~\dfrac{1}{2}\leqslant x <1\\
   0 ~~~~~~~~~~~~~~~\text{Otherwise}
   \end{cases}
   $$
   Then value of c is equal to ...
   \\
\solution
\input{solutions/ma/2016/10/Assignment5.tex}
%
\item Let $A_{1},A_{2},.....A_{n}$ be n independent events in which the Probability of occurence of the event $A_{i}$ is given by P($A_{i}$) = 1 - $\frac{1}{\alpha^i}$, $\alpha >1$, i = 1,2,3,....n.Then the probability that atleast one of the events occurs is
\begin{enumerate}
    \item  1 - $\frac{1}{\alpha^\frac{n(n+1)}{2}}$ \hspace{0.95cm}
    \item  $\frac{1}{\alpha^\frac{n(n+1)}{2}}$\hspace{1.5cm}
    \\ \\
    \item  $\frac{1}{\alpha^n}$ \hspace{2.15cm}
    \item 1 - $\frac{1}{\alpha^n}$\hspace{0.95cm}
  \end{enumerate}
  %
  \solution
  \input{solutions/ma/2005/25.tex}
  %
  \item Let the random variable X have the distribution function 
$F(x)= \begin{cases}
       0  & if \:x<0\\
       \frac{x}{2} & if\: 0 \le x <1\\
       \frac{3}{5} & if \:1 \le x <2\\
       \frac{1}{2} +\frac{x}{8} & if\: 2\le x <3\\
       1  & if\: x\ge 3
    \end{cases}$\\
Then $\pr{2\le x <4}$ is equal to 
%
\\
  \solution
  \input{solutions/ma/2015/9/ASSIGNMENT_5.tex}
\item Let Z be the vertical coordinate, between -1 and 1, of a point chosen uniformly at random on the
\begin{math}
\text{surface of a unit sphere in }R^3.\text{ Then,} \pr{\frac{-1}{2} \leq Z \leq \frac{1}{2}}
\end{math}
is
\\
  \solution
  \input{solutions/ma/2006/67.tex}
%
\item Let $X_1$ and $X_2$ be independent geometric random variables with the same probability
mass function given by $\pr{X = k} = p(1-p)^{k-1}$
, $k = 1, 2, \ldots$ Then the value of
$\pr{X_1 = 2 | X_1 + X_2 = 4}$ correct up to three decimal places is
\\
  \solution
  \input{solutions/ma/2018/54.tex}
%
\item     Let X and Y have joint probability function given by\\
    $$f_{X,Y}(x,y)=\left\{
    \begin{array}{ll}
      2 & 0\leq x\leq 1-y,0\leq y\leq 1 \\
      0 & otherwise \\
    \end{array} 
    \right. $$
    If $f_{Y}$ denotes the marginal probability density function of Y, then $f_{Y}(1/2)=?$
\\
\solution

    Let the cumulative distribution function of the random variable X be given by 
    $$F_{X}(x)=\left\{
    \begin{array}{ll}
      0 & x<0 \\
      x & 0\leq x<1/2\\
      (1+x)/2 & 1/2\leq x <1\\
      1 & x\geq 1
    \end{array} 
    \right. $$
    Then $\pr{X=1/2}=?$

Given,
$$F_{X}(x)=\left\{
    \begin{array}{ll}
      0 & x<0 \\
      x & 0\leq x<1/2\\
      (1+x)/2 & 1/2\leq x <1\\
      1 & x\geq 1
    \end{array} 
    \right. $$
\begin{align}
\tag{24.1}
\pr{X=1/2}=\pr{X\leq 1/2}-\pr{X<1/2}
\end{align}
\begin{align}
\tag{24.2}
\implies \pr{X=1/2}=F_{X}\brak{\frac{1}{2}}-F_{X}\brak{\frac{1}{2}^-}\\
\tag{24.3}
\implies \pr{X=1/2}=\brak{1+1/2}/2-1/2
\end{align}
\begin{align}
\tag{24.4}
\therefore \pr{X=1/2}=1/4
\end{align}
\newpage
\begin{figure}[t!]
\centering
\includegraphics[width=8cm]{cdf_plot.png}
\caption{cdf of Random variable X}
\end{figure}


%
\item Let X be a standard normal random variable. Then $\pr{X < 0 |\;\abs{\lfloor X\rfloor} = 1}$ is equal to
\begin{enumerate}[label = \alph*)]
    \item $\cfrac{\Phi(1) - \frac{1}{2}}{\Phi(2) - \frac{1}{2}}$
    \item $\cfrac{\Phi(1) + \frac{1}{2}}{\Phi(2) + \frac{1}{2}}$
    \item $\cfrac{\Phi(1) - \frac{1}{2}}{\Phi(2) + \frac{1}{2}}$
    \item $\cfrac{\Phi(1) + 1}{\Phi(2) + 1}$
\end{enumerate}
%
\solution
\input{solutions/ma/2016/49.tex}
%
\item Let $X$ be a random variable with probability mass function 
$p(n) = \brak{\frac{1}{4}}\brak{\frac{3}{4}}^{n-1}  n=1,2 \ldots $ \\
Then $E[X-3|X>3]$ is \dots
\\
\solution
\input{solutions/ma/2017/46/Assignment-5.tex}
%
\item Let (X,Y) be a random vector such that, for any $y>0$, the conditional probability density function of X given $Y=y$ is $$f_{X|Y=y}(x)=ye^{-yx} \:,x>0. $$ If the marginal probability density function of Y is $$g(y)=ye^{-y}\:,y>0$$ then $E(Y|x=1)=$
\\
\solution
\input{solutions/st/2020/43.tex}

\end{enumerate}
\end{document}
