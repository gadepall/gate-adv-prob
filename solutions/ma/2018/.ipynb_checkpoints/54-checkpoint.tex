Let 
\begin{align}
\label{eq:dice_pmf_xi}
p_{X_i}(k)=\pr{X_i=k}=
\begin{cases}
p(1-p)^{k-1} & n=1,2,...
\\
0 & otherwise
\end{cases}
\end{align}
where i=1,2
\begin{align}
\pr{A|B}=\frac{\pr{AB}}{\pr{B}}
\end{align}
\begin{align}
(X_1 = 2) \cap (X_1 + X_2 = 4)=\brak{X_1=2,X_2=2}
\end{align}
Thus,
\begin{align}
  \pr{X_1 = 2 | X_1 + X_2 = 4}&=\frac{\pr{X_1=2,X_2=2}}{\pr{X_1+X_2=4}}
 \end{align}
 Since the two events are independent,
\begin{align}\label{result}
  \pr{X_1 = 2 | X_1 + X_2 = 4}=\frac{\pr{X_1=2}\pr{X_2=2}}{\pr{X_1+X_2=4}}
 \end{align}
Let
\begin{align}\label{eq:dice_xdef}
X=X_1+X_2
\end{align}
From \eqref{eq:dice_xdef},
\begin{align}
p_X(n) &= \pr{X_1 + X_2 = n} = \pr{X_1  = n -X_2}
\\
&= \sum_{k}^{}\pr{X_1  = n -k | X_2 = k}p_{X_2}(k)
\label{eq:dice_x_sum}
\end{align}
after unconditioning.  $\because X_1$ and $X_2$ are independent,
\begin{multline}
\pr{X_1  = n -k | X_2 = k} 
\\
= \pr{X_1  = n -k}
= p_{X_1}(n-k)
\label{eq:dice_x1_indep}
\end{multline}
From \eqref{eq:dice_x_sum} and \eqref{eq:dice_x1_indep},
\begin{align}
p_X(n) = \sum_{k}^{}p_{X_1}(n-k)p_{X_2}(k) = p_{X_1}(n)*p_{X_2}(n)
\label{eq:dice_x_conv}
\end{align}
where $*$ denotes the convolution operation. 
%\cite{proakis_dsp}.  
Substituting from \eqref{eq:dice_pmf_xi}
in \eqref{eq:dice_x_conv},
\begin{align}
p_X(n)& = \sum_{k=1}^{n-1}p_{X_1}(n-k)p_{X_2}(k)\\
& = \sum_{k=1}^{n-1} (1-p)^{k-1} p \cdot (1-p)^{n-k-1} p \\ & = (1-p)^{n-2} p^2 \sum_{k=1}^{n-1} 1 \\& = (n-1) (1-p)^{n-2}p^2\label{ref}\end{align}
From \eqref{ref} and \eqref{eq:dice_pmf_xi} we have
\begin{align}
&\pr{X_1=2}=\pr{X_2=2}=p(1-p)\\
&\pr{X_1+X_2=4}=3(1-p)^2p^2
\end{align}
Substituting in \eqref{result}
\begin{align}
 \pr{X_1 = 2 | X_1 + X_2 = 4}
 &=\frac{(1-p)^2p^2}{3(1-p)^2p^2}\\
 &=\frac{1}{3}
\end{align}