
\begin{align}
    &\abs{\lfloor X\rfloor} = 1\\
    \implies &\lfloor X\rfloor = 1\; or\; -1\\
    \implies &X \in [1, 2) \cup [-1, 0)
\end{align}
Here 
\begin{align*}
    \lfloor X\rfloor = greatest\; integer\; less\; than\; or\; equal\; to\; X
\end{align*}
Thus required probability
\begin{align}
    &= \frac{\pr{X\in [-1,0)}}{\pr{X \in [1, 2) \cup [-1, 0)}}
\end{align} 
Using symmetry of standard normal random variable about y = 0, we have required probability 
\begin{align}
    &= \cfrac{\pr{X\in (0,1]}}{\pr{X \in [1, 2) \cup (0, 1])}}\\
    &= \cfrac{\pr{X \in (0, 1]}}{\pr{X \in (0, 2)}}\\
    &= \cfrac{\pr{X < 1} - \pr{X < 0}}{\pr{X < 2} - \pr{X < 0}}\\
    &= \cfrac{\Phi(1) - \Phi(0)}{\Phi(2) - \Phi(0)}\\
    &= \cfrac{\Phi(1) - \frac{1}{2}}{\Phi(2) - \frac{1}{2}}\label{bato}\\
    &= \cfrac{0.841 - 0.5}{0.977 - 0.5}\label{gato}\\
    &= 0.715
\end{align}
Here $\Phi(x)$ represents the standard normal cumulative density function. Thus 
\begin{align}
    X \sim \mathcal{0}{1}
\end{align}
and 
\begin{align}
    \Phi(x) = \int_{-\infty}^x f_X(x)dx
\end{align}
It can easily be seen that $\Phi(0) = \frac{1}{2}$, which has been used to obtain \eqref{bato}.
\eqref{gato} was obtained by consulting tables for $\Phi(x)$