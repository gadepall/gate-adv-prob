\documentclass[journal,12pt,twocolumn]{IEEEtran}
%
\usepackage{setspace}
\usepackage{gensymb}
%\doublespacing
\singlespacing

%\usepackage{graphicx}
%\usepackage{amssymb}
%\usepackage{relsize}
\usepackage[cmex10]{amsmath}
\usepackage{siunitx}
%\usepackage{amsthm}
%\interdisplaylinepenalty=2500
%\savesymbol{iint}
%\usepackage{txfonts}
%\restoresymbol{TXF}{iint}
%\usepackage{wasysym}
\usepackage{amsthm}
\usepackage{iithtlc}
\usepackage{mathrsfs}
\usepackage{txfonts}
\usepackage{stfloats}
\usepackage{steinmetz}
%\usepackage{bm}
\usepackage{cite}
\usepackage{cases}
\usepackage{subfig}
%\usepackage{xtab}
\usepackage{longtable}
\usepackage{multirow}
%\usepackage{algorithm}
%\usepackage{algpseudocode}
\usepackage{enumitem}
\usepackage{mathtools}
\usepackage{tikz}
\usepackage{circuitikz}
\usepackage{pgfplots}
\usepackage{verbatim}
\usepackage{tfrupee}
\usepackage[breaklinks=true]{hyperref}
%\usepackage{stmaryrd}
\usepackage{tkz-euclide} % loads  TikZ and tkz-base
%\usetkzobj{all}
\usetikzlibrary{calc,math}
\usetikzlibrary{fadings}
\usepackage{listings}
    \usepackage{color}                                            %%
    \usepackage{array}                                            %%
    \usepackage{longtable}                                        %%
    \usepackage{calc}                                             %%
    \usepackage{multirow}                                         %%
    \usepackage{hhline}                                           %%
    \usepackage{ifthen}                                           %%
  %optionally (for landscape tables embedded in another document): %%
    \usepackage{lscape}     
\usepackage{multicol}
\usepackage{chngcntr}
%\usepackage{enumerate}

%\usepackage{wasysym}
%\newcounter{MYtempeqncnt}
\DeclareMathOperator*{\Res}{Res}
%\renewcommand{\baselinestretch}{2}
\renewcommand\thesection{\arabic{section}}
\renewcommand\thesubsection{\thesection.\arabic{subsection}}
\renewcommand\thesubsubsection{\thesubsection.\arabic{subsubsection}}

\renewcommand\thesectiondis{\arabic{section}}
\renewcommand\thesubsectiondis{\thesectiondis.\arabic{subsection}}
\renewcommand\thesubsubsectiondis{\thesubsectiondis.\arabic{subsubsection}}

% correct bad hyphenation here
\hyphenation{op-tical net-works semi-conduc-tor}
\def\inputGnumericTable{}                                 %%

\lstset{
%language=C,
frame=single, 
breaklines=true,
columns=fullflexible
}
%\lstset{
%language=tex,
%frame=single, 
%breaklines=true
%}

\begin{document}
%


\newtheorem{theorem}{Theorem}[section]
\newtheorem{problem}{Problem}
\newtheorem{proposition}{Proposition}[section]
\newtheorem{lemma}{Lemma}[section]
\newtheorem{corollary}[theorem]{Corollary}
\newtheorem{example}{Example}[section]
\newtheorem{definition}[problem]{Definition}
%\newtheorem{thm}{Theorem}[section] 
%\newtheorem{defn}[thm]{Definition}
%\newtheorem{algorithm}{Algorithm}[section]
%\newtheorem{cor}{Corollary}
\newcommand{\BEQA}{\begin{eqnarray}}
\newcommand{\EEQA}{\end{eqnarray}}
\newcommand{\define}{\stackrel{\triangle}{=}}

\bibliographystyle{IEEEtran}
%\bibliographystyle{ieeetr}


\providecommand{\mbf}{\mathbf}
\providecommand{\pr}[1]{\ensuremath{\Pr\left(#1\right)}}
\providecommand{\qfunc}[1]{\ensuremath{Q\left(#1\right)}}
\providecommand{\sbrak}[1]{\ensuremath{{}\left[#1\right]}}
\providecommand{\lsbrak}[1]{\ensuremath{{}\left[#1\right.}}
\providecommand{\rsbrak}[1]{\ensuremath{{}\left.#1\right]}}
\providecommand{\brak}[1]{\ensuremath{\left(#1\right)}}
\providecommand{\lbrak}[1]{\ensuremath{\left(#1\right.}}
\providecommand{\rbrak}[1]{\ensuremath{\left.#1\right)}}
\providecommand{\cbrak}[1]{\ensuremath{\left\{#1\right\}}}
\providecommand{\lcbrak}[1]{\ensuremath{\left\{#1\right.}}
\providecommand{\rcbrak}[1]{\ensuremath{\left.#1\right\}}}
\theoremstyle{remark}
\newtheorem{rem}{Remark}
\newcommand{\sgn}{\mathop{\mathrm{sgn}}}
\providecommand{\abs}[1]{\left\vert#1\right\vert}
\providecommand{\res}[1]{\Res\displaylimits_{#1}} 
\providecommand{\norm}[1]{\left\lVert#1\right\rVert}
%\providecommand{\norm}[1]{\lVert#1\rVert}
\providecommand{\mtx}[1]{\mathbf{#1}}
\providecommand{\mean}[1]{E\left[ #1 \right]}
\providecommand{\fourier}{\overset{\mathcal{F}}{ \rightleftharpoons}}
%\providecommand{\hilbert}{\overset{\mathcal{H}}{ \rightleftharpoons}}
\providecommand{\ztrans}{\overset{\mathcal{Z}}{ \rightleftharpoons}}
\providecommand{\system}{\overset{\mathcal{H}}{ \longleftrightarrow}}
	%\newcommand{\solution}[2]{\textbf{Solution:}{#1}}
\newcommand{\solution}{\noindent \textbf{Solution: }}
\newcommand{\cosec}{\,\text{cosec}\,}
\providecommand{\dec}[2]{\ensuremath{\overset{#1}{\underset{#2}{\gtrless}}}}
\newcommand{\myvec}[1]{\ensuremath{\begin{pmatrix}#1\end{pmatrix}}}
\newcommand{\mydet}[1]{\ensuremath{\begin{vmatrix}#1\end{vmatrix}}}
\providecommand{\gauss}[2]{\mathcal{N}\ensuremath{\left(#1,#2\right)}}
%\providecommand{\system}[1]{\overset{\mathcal{#1}}{ \longleftrightarrow}}
\newcommand*{\permcomb}[4][0mu]{{{}^{#3}\mkern#1#2_{#4}}}
\newcommand*{\perm}[1][-3mu]{\permcomb[#1]{P}}
\newcommand*{\comb}[1][-1mu]{\permcomb[#1]{C}}

\title{
%\logo{
GATE Problems in Probability
%}
}

\maketitle

\begin{abstract}
These problems have been selected from GATE question papers and can be used for conducting tutorials in courses related to a first course in probability.
\end{abstract}
%\centering \textbf{\Large Probability}\\

\begin{enumerate}
\setlength\itemsep{2em}

\item Let X be a random variable with the following cumulative distribution function:
\begin{align}
F\brak{x} 
= 
\begin{cases}
0           & x < 0 \\
x^2         & 0 \leq x < \frac{1}{2} \\
\frac{3}{4} & \frac{1}{2} \leq x < 1 \\
1           &  x \geq 1 
\end{cases}
\end{align}
Then $ P\brak{\frac{1}{4} < X < 1}$ is equal to

%
\solution
\input{solutions/ma/2016/30/Assignment4.tex}
\item Let X and Y be continuous random variables with the joint probability density function 
\begin{align}
f\brak{x,y}= 
\begin{cases}
ae^{-2y} & 0<x<y<\infty \\
0 & \text{otherwise}.
\end{cases}   
\end{align}
Then $E\brak{X|Y=2}$ is \dots
\solution
\input{solutions/ma/2010/49.tex}
%
\item A continuous random variable X has the probability density function
\begin{align*}
    f(x) &= \begin{cases} 
      \frac{3}{5}e^{-\frac{3}{5}x} & x > 0 \\
      0 & x\leq 0\\
   \end{cases} 
\end{align*} 
The probability density function of $Y=3X+2$  is
\begin{enumerate}
    \item \begin{align*}
         f(y) &= \begin{cases} 
      \frac{1}{5}e^{-\frac{1}{5} (y-2)} & y > 2 \\
      0 & y \leq 2\\
   \end{cases} 
    \end{align*}
\item \begin{align*}
    f(y) &= \begin{cases} 
      \frac{2}{5}e^{-\frac{2}{5} (y-2)} & y > 2 \\
      0 & y \leq 2\\
   \end{cases} 
\end{align*} 
\item  \begin{align*}
    f(y) &= \begin{cases} 
      \frac{3}{5}e^{-\frac{3}{5} (y-2)} & y > 2 \\
      0 & y \leq 2\\
   \end{cases} 
\end{align*} 
\item \begin{align*}
    f(y) &= \begin{cases} 
      \frac{4}{5}e^{-\frac{4}{5} (y-2)} & y > 2 \\
      0 & y \leq 2\\
   \end{cases} 
\end{align*} 
\end{enumerate}
%
\solution
\input{solutions/ma/2012/8.tex}
%
\item    Let the probability density function of a random variable X be
   $$
   f(x)=
   \begin{cases}
   x~~~~~~~~~~~~~~~0\leqslant x < \dfrac{1}{2}\\
   c(2x-1)^2~~~~~\dfrac{1}{2}\leqslant x <1\\
   0 ~~~~~~~~~~~~~~~\text{Otherwise}
   \end{cases}
   $$
   Then value of c is equal to ...
   \\
\solution
\input{solutions/ma/2016/10/Assignment5.tex}
%
\item Let $A_{1},A_{2},.....A_{n}$ be n independent events in which the Probability of occurence of the event $A_{i}$ is given by P($A_{i}$) = 1 - $\frac{1}{\alpha^i}$, $\alpha >1$, i = 1,2,3,....n.Then the probability that atleast one of the events occurs is
\begin{enumerate}
    \item  1 - $\frac{1}{\alpha^\frac{n(n+1)}{2}}$ \hspace{0.95cm}
    \item  $\frac{1}{\alpha^\frac{n(n+1)}{2}}$\hspace{1.5cm}
    \\ \\
    \item  $\frac{1}{\alpha^n}$ \hspace{2.15cm}
    \item 1 - $\frac{1}{\alpha^n}$\hspace{0.95cm}
  \end{enumerate}
  %
  \solution
  \input{solutions/ma/2005/25.tex}
  %
  \item Let the random variable X have the distribution function 
$F(x)= \begin{cases}
       0  & if \:x<0\\
       \frac{x}{2} & if\: 0 \le x <1\\
       \frac{3}{5} & if \:1 \le x <2\\
       \frac{1}{2} +\frac{x}{8} & if\: 2\le x <3\\
       1  & if\: x\ge 3
    \end{cases}$\\
Then $\pr{2\le x <4}$ is equal to 
%
\\
  \solution
  \input{solutions/ma/2015/9/ASSIGNMENT_5.tex}
\item Let Z be the vertical coordinate, between -1 and 1, of a point chosen uniformly at random on the
\begin{math}
\text{surface of a unit sphere in }R^3.\text{ Then,} \pr{\frac{-1}{2} \leq Z \leq \frac{1}{2}}
\end{math}
is
\\
  \solution
  \input{solutions/ma/2006/67.tex}
%
\item Let $X_1$ and $X_2$ be independent geometric random variables with the same probability
mass function given by $\pr{X = k} = p(1-p)^{k-1}$
, $k = 1, 2, \ldots$ Then the value of
$\pr{X_1 = 2 | X_1 + X_2 = 4}$ correct up to three decimal places is
\\
  \solution
  \input{solutions/ma/2018/54.tex}
%
\item     Let X and Y have joint probability function given by\\
    $$f_{X,Y}(x,y)=\left\{
    \begin{array}{ll}
      2 & 0\leq x\leq 1-y,0\leq y\leq 1 \\
      0 & otherwise \\
    \end{array} 
    \right. $$
    If $f_{Y}$ denotes the marginal probability density function of Y, then $f_{Y}(1/2)=?$
\\
\solution

    Let the cumulative distribution function of the random variable X be given by 
    $$F_{X}(x)=\left\{
    \begin{array}{ll}
      0 & x<0 \\
      x & 0\leq x<1/2\\
      (1+x)/2 & 1/2\leq x <1\\
      1 & x\geq 1
    \end{array} 
    \right. $$
    Then $\pr{X=1/2}=?$

Given,
$$F_{X}(x)=\left\{
    \begin{array}{ll}
      0 & x<0 \\
      x & 0\leq x<1/2\\
      (1+x)/2 & 1/2\leq x <1\\
      1 & x\geq 1
    \end{array} 
    \right. $$
\begin{align}
\tag{24.1}
\pr{X=1/2}=\pr{X\leq 1/2}-\pr{X<1/2}
\end{align}
\begin{align}
\tag{24.2}
\implies \pr{X=1/2}=F_{X}\brak{\frac{1}{2}}-F_{X}\brak{\frac{1}{2}^-}\\
\tag{24.3}
\implies \pr{X=1/2}=\brak{1+1/2}/2-1/2
\end{align}
\begin{align}
\tag{24.4}
\therefore \pr{X=1/2}=1/4
\end{align}
\newpage
\begin{figure}[t!]
\centering
\includegraphics[width=8cm]{cdf_plot.png}
\caption{cdf of Random variable X}
\end{figure}


%
\item Let X be a standard normal random variable. Then $\pr{X < 0 |\;\abs{\lfloor X\rfloor} = 1}$ is equal to
\begin{enumerate}[label = \alph*)]
    \item $\cfrac{\Phi(1) - \frac{1}{2}}{\Phi(2) - \frac{1}{2}}$
    \item $\cfrac{\Phi(1) + \frac{1}{2}}{\Phi(2) + \frac{1}{2}}$
    \item $\cfrac{\Phi(1) - \frac{1}{2}}{\Phi(2) + \frac{1}{2}}$
    \item $\cfrac{\Phi(1) + 1}{\Phi(2) + 1}$
\end{enumerate}
%
\solution
\input{solutions/ma/2016/49.tex}
%
\item Let $X$ be a random variable with probability mass function 
$p(n) = \brak{\frac{1}{4}}\brak{\frac{3}{4}}^{n-1}  n=1,2 \ldots $ \\
Then $E[X-3|X>3]$ is \dots
\\
\solution
\input{solutions/ma/2017/46/Assignment-5.tex}
%
\item Let (X,Y) be a random vector such that, for any $y>0$, the conditional probability density function of X given $Y=y$ is $$f_{X|Y=y}(x)=ye^{-yx} \:,x>0. $$ If the marginal probability density function of Y is $$g(y)=ye^{-y}\:,y>0$$ then $E(Y|x=1)=$
\\
\solution
\input{solutions/st/2020/43.tex}
%
\item Let $X$ and $Y$ be jointly distributed random
variables having the joint probability
density function
\[
f(x,y) = \begin{cases}
            \frac{1}{\pi}, &\text{if}\quad x^2 + y^2 \leq 1\\
             0, &\text{otherwise}\\
            \end{cases}
\]
Then $\pr{Y > \text{max}(X,-X)}$ is
\\
\solution 
\input{solutions/ma/2007/14.tex}
%
\item Let $X$ and $Y$ be two continuous random variables with the joint probability density function
\begin{align}
    f(x,y) =
    \begin{cases}
    2, & 0<x+y<1, x>0, y>0,\\
    0, & elsewhere.
    \end{cases}
\end{align}
$E\brak{X \bigm| Y=\frac{1}{2}}$ is
\begin{enumerate}
    \item $1/4$
    \item $1/2$
    \item $1$
    \item $2$
\end{enumerate}
\solution
%\input{solutions/ma/2011/49.tex}
%
\item An urn contains four balls, each ball having equal probability of being white or black. Three black balls are added to the urn. The probability that five balls in the urn are black is
\\
\solution
\input{solutions/ma/2018/11.tex}
%
\item There are five bags each containing identical sets of ten distinct chocolates. One chocolate is picked from each bag.\\
The probability that at least two chocolates are identical is
%
\\
\solution
\input{solutions/ma/2021/8.tex}
%
Consider the trinomial distribution with the probability mass function 
\begin{multline}
    \nonumber \pr{X=x,Y=y}\\=\brak{\frac{7!}{x!y!(7-x-y)!}}(0.6)^x(0.2)^y(0.2)^{7-x-y}
\end{multline}
where $x\geq0 , y\geq0 \;and\; {x+y}\leq7$.
Then $E\brak{Y|X=3}$ is equal to
%
\\
\solution

Probability mass function of a trinomial  distribution is :
\begin{multline}
   \p{x}{y} \\=\brak{\frac{7!}{x!y!(7-x-y)!}}(0.6)^x(0.2)^y(0.2)^{7-x-y}\\
  \nonumber  =\brak{\frac{7!}{x!(7-x)!}\frac{(7-x)!}{y!(7-x-y)!}}(0.6)^x(0.2)^y(0.2)^{7-x-y}
\end{multline}
\begin{equation}
    \p{x}{y}=\comb{7}{x}\comb{7-x}{y}(0.6)^x(0.2)^y(0.2)^{7-x-y}\label{1}
\end{equation}
Using \eqref{1}, $\q{x}$ is 
\begin{align}
   \nonumber \q{x}&=\sum_{y=0}^{7-x} \p{x}{y}\\
  \nonumber &=\comb{7}{x}(0.6)^x\,\sum_{y=0}^{7-x}\, \comb{7-x}{y} (0.2)^y(0.2)^{7-x-y} \\
  \nonumber  &=\comb{7}{x}(0.6)^x\,{(0.4)^{7-x}}\\
    \q{x}&=\comb{7}{x}(0.6)^x\,{(0.4)^{7-x}}\label{2}
\end{align}
We have to find $\mean{Y|X=3}$ ,
\begin{align}
    \mean{Y|X=3}&=\sum_{y=0}^4 \: y P(Y=y|X=3)\\
    \mean{Y|X=3}&=\sum_{y=0}^4 y \brak{  \frac{ \p{3}{y}}{P(X=3)} }\label{3}
\end{align}
By taking X=3 in \eqref{1} and \eqref{2}  to use in \eqref{3},
\begin{align}
   \nonumber \mean{Y|X=3}&=\sum_{y=0}^4 y \brak{ \frac{ \p{3}{y}}{P(X=3)}}\\
  \nonumber  &=\sum_{y=0}^4 y   \brak{\frac{\comb{7}{3}\comb{4}{y}(0.6)^3(0.2)^y(0.2)^{4-y}}{\comb{7}{3}(0.6)^3\,{(0.4)^{4}}}}\\
 \nonumber &=\sum_{y=0}^4 y   \brak{\frac{\comb{4}{y}(0.2)^{4}}{{(0.4)^{4}}}}\\
 \mean{Y|X=3}&=\sum_{y=0}^4 {\frac{y(\comb{4}{y})}{{16}}}\label{4}
 \end{align}
 We know that,
 \begin{align}
     \comb{n}{r} = \frac{n}{r}\brak{\comb{n-1}{r-1}}\label{5}
 \end{align}
 Using \eqref{5} in \eqref{4},
 \begin{align}
  \mean{Y|X=3}&=\frac{1}{16}\sum_{y=0}^4\, {y(\comb{4}{y})}\\
  &=\frac{1}{16}\sum_{y=1}^{4} y\brak{\frac{4}{y}}(\comb{3}{y-1})\\
  &=\frac{1}{4}\sum_{k=0}^{3}(\comb{3}{k})\\
  &=\frac{1}{4}(1+1)^3=\frac{1}{4}(8)\\
  \mean{Y|X=3}&=2
\end{align}
Therefore the value of $\mean{Y|X=3}=2$.
\item Let E and F be any two events with $P(E \cup F) = 0.8$, $P(E) = 0.4$ and $P(E|F) = 0.3$ then P(F) is \newline
\begin{enumerate}
\item $\frac{3}{7}$ 
\item $\frac{4}{7}$ 
\item $\frac{3}{5}$ 
\item $\frac{2}{5}$ 
\end{enumerate}
%
\solution
\input{solutions/ma/2010/1.tex}
%
\item The number $N$ of persons getting injured in a bomb blast at a busy market place is a random variable having a Poisson Distribution with parameter $\lambda(\geq 1)$.
 A person injured in the explosion may either suffer a minor injury requiring first aid or suffer a major injury requiring hospitalisation. Let the number of persons with minor injury be $N_1$ and the conditional distribution of $N_1$ given N is
 \begin{align}
     \pr{N_1 = i \vert N} = \frac{1}{N}
     \label{condition}
 \end{align}
 Find the expected number of persons requiring hospitalisation.
 %
\solution
\input{solutions/ma/2001/18.tex}
 %
 \item The time to failure, in months, of lights bulbs \\manufactured at two plants A and B
obey the exponential distributions with means 6 and 2 months respectively. Plant B produces
four times as many bulbs as plant A does. Bulbs from these two plants are indistinguishable.
They are mixed and sold together. Given that a bulb purchased at random is working after 12 months, What is the probability that it was manufactured in plant A?
\\
\solution
\input{solutions/ma/2014/36.tex}
 %
\item The lifetime of two brands of bulbs X and Y are exponentially distributed with the mean life of 100 hours. Bulb X is switched on 15 hours after bulb Y has been switched on. The probability that bulb X fails before bulb Y is 
\begin{enumerate}[label=(\Alph*)]
    \item $\frac{15}{100}$ \\
    \item $\frac{1}{2}$ \\
    \item $\frac{85}{100}$ \\
    \item 0 
  \end{enumerate}
  \solution
  \input{solutions/ma/2005/26.tex}
  %
  \item Let $X_{1},X_{2},\dots$, be a sequence of independent and identically distributed random variables with $P(X_{1}=1)=\dfrac{1}{4}$ and $P(X_{1}=2)=\dfrac{3}{4}$. If $\bar X_{n}=\dfrac{1}{n}\displaystyle\sum_{i=1}^{n}X_{i}$,  for $n=1,2,\dots$, then $\displaystyle\lim_{n\to\infty}P(\bar X_{n} \leq 1.8)$ is equal to
  \\
  \solution
  \input{solutions/ma/2015/32.tex}
  %
  \item Let $X$ be the number of heads in 4 tosses of a fair coin by Person 1 and let $Y$ be the number of heads in 4 tosses of a fair coin by Person 2. Assume that all the tosses are independent. Then the value of $\Pr{(X=Y)}$ correct up to three decimal places is $\rule{1.6cm}{0.15mm}$ .
%
  \solution
  \input{solutions/ma/2018/53.tex}
\item The probability density function of a random variable X is
\begin{equation}
f(x)=
\begin{cases}
\frac{1}{\lambda}e^{\brak{-\frac{x}{\lambda}}}, & x>0\\
0, & x\leq 0
\end{cases}
\end{equation}
where $\lambda>0.$ For testing the hypothesis $H_{0}:\lambda=3$ against $H_{1}:\lambda=5$, a test is given as "Reject $H_0$ if $X\geq 4.5$".The probability of type 1 error and power of the test are respectively: 
\begin{enumerate}
\begin{multicols}{2}
\setlength\itemsep{1em}
\item 0.1353 and 0.4966\\
\item 0.1827 and 0.379\\
\item 0.2021 and 0.4493\\
\item 0.2231 and 0.4066
\end{multicols}
\end{enumerate}
%
  \solution
  \input{solutions/ma/2012/30.tex}
  %
  \item  Let $E$ and $F$ be any two events with $\pr{E}=0.4, \pr{F}=0.3$
and $\pr{F|E}=3 \pr{F|E'}$. Then $\pr{E|F}$ equals ......
\\
  \solution
  \input{solutions/ma/2017/50.tex}
%
\item Let a random variable $X$ follow the exponential distribution with mean 2. Define $Y$ such that:
\begin{align}
   Y =  \left[X-2\right|X>2]\nonumber
\end{align}
Then $E(Y)$ is equal to:\\
\\(A) $\frac{1}{4}$\\
\\(B) $\frac{1}{2}$\\
\\(C) 1\\
\\(D) 2\\
\item If A and B are two events and the probability $\Pr(B) \neq 1$,then\\
\begin{equation}
    \frac{\Pr(A)-\Pr(A \cap B)}{1-\Pr{(B)}}
\end{equation}
equals
\begin{enumerate}
\begin{multicols}{2}
\setlength\itemsep{1em}
\item $\Pr{(A|\bar{B})}$\\
\item $\Pr{(A|B)}$\\
\item $\Pr{(\bar{A}|B)}$\\
\item $\Pr{(\bar{A}|\bar{B})}$
\end{multicols}
\end{enumerate}
  \solution
  \input{solutions/ma/1997/18.tex}
%
\item If a random variable X assumes only positive integral values ,with the probability
\begin{equation}
 P(X=x)=\frac{2}{3}\brak{\frac{1}{3}}^{x-1},x=1,2,3,....,   
\end{equation}
then $E(X)$ is 
\begin{enumerate}
\begin{multicols}{2}
\item $ \frac{2}{9}$\\
\item $\frac{2}{3}$\\
\item $ 1$\\
\item $\frac{3}{2}$
\end{multicols}
\end{enumerate}
  \solution
  \input{solutions/ma/2012/29.tex}
%
\item Let $X ,Y$ be continuous random variables with joint density function
\begin{align*}
    f_{X,Y}(x,y)=\begin{cases}
    e^{-y}(1-e^{-x}) \text{   if } 0< x<y<\infty\\
    e^{-x}(1-e^{-y}) \text{   if } 0< y\leq x<\infty
    \end{cases}
\end{align*}
Then The value of $E[X+Y]$ is 
%
\solution
  \input{solutions/ma/2014/37.tex}
%
\item Suppose customers arrive at an ATM facility according to Poisson process with rate 5 customers per hour. The probability (rounded off to two decimal places) that no customer arrives at the ATM facility from 1:00pm to 1:18pm.
%
\solution
  \input{solutions/st/2019/48.tex}
%
\item     Let the cumulative distribution function of the random variable X be given by 
    $$F_{X}(x)=\left\{
    \begin{array}{ll}
      0 & x<0 \\
      x & 0\leq x<1/2\\
      (1+x)/2 & 1/2\leq x <1\\
      1 & x\geq 1
    \end{array} 
    \right. $$
    Then $\pr{X=1/2}=?$
%
\solution
  
    Let the cumulative distribution function of the random variable X be given by 
    $$F_{X}(x)=\left\{
    \begin{array}{ll}
      0 & x<0 \\
      x & 0\leq x<1/2\\
      (1+x)/2 & 1/2\leq x <1\\
      1 & x\geq 1
    \end{array} 
    \right. $$
    Then $\pr{X=1/2}=?$

Given,
$$F_{X}(x)=\left\{
    \begin{array}{ll}
      0 & x<0 \\
      x & 0\leq x<1/2\\
      (1+x)/2 & 1/2\leq x <1\\
      1 & x\geq 1
    \end{array} 
    \right. $$
\begin{align}
\tag{24.1}
\pr{X=1/2}=\pr{X\leq 1/2}-\pr{X<1/2}
\end{align}
\begin{align}
\tag{24.2}
\implies \pr{X=1/2}=F_{X}\brak{\frac{1}{2}}-F_{X}\brak{\frac{1}{2}^-}\\
\tag{24.3}
\implies \pr{X=1/2}=\brak{1+1/2}/2-1/2
\end{align}
\begin{align}
\tag{24.4}
\therefore \pr{X=1/2}=1/4
\end{align}
\newpage
\begin{figure}[t!]
\centering
\includegraphics[width=8cm]{cdf_plot.png}
\caption{cdf of Random variable X}
\end{figure}


%
\item Let A and b be two events such that $\pr{B}=\frac{3}{4}$ and $\pr{A + B^{\prime}}=\frac{1}{2}$.If A and B are independent, then $\pr{A}$ equals
\\
\solution
  \input{solutions/st/2021/14.tex}
\item Let $(X,Y)$ have a bivariate normal distribution with the joint probability density function
\begin{align}
f_{X,Y}(x,y)=\frac{1}{\pi}e^{(\frac{3}{2}xy-\frac{25}{32}x^2-2y^2)}\\
-\infty < x,y < \infty
\end{align}
Then $E(XY)$ equals 
%
\\
\solution
  \input{solutions/st/2021/44.tex}
%
\item Let $X_{1}$ be	an	exponential	random	variable with mean 1 and $X_{2}$ a gamma	random variable	with mean 2	and	variance 2.	If $X_{1}$ and $X_{2}$ are independently	distributed, then $\pr{X_1 < X_2}$ is equal	to ..........	
\\
\solution
  \input{solutions/ma/2016/47.tex}
%
\item Let $\Omega = (\,0,1]\,$ be the sample space and let $P\brak{.}$ be a probability distribution given by\\
\begin{align}
P\brak{(\,0,x]\,} = 
\begin{cases}
   \frac{x}{2} & 0 \le x < \frac{1}{2}\\
   x & \frac{1}{2} \le x \le 1
\end{cases}
\end{align}
Find $P\brak{\frac{1}{2}}$
\\
\solution
  \input{solutions/ma/2015/27/latex4.tex}
%
\item Let $(X,Y)$ be a two-dimensional random variable such that $E(X)=E(Y)=1/2$, $Var(X)=Var(Y)=1$ and $Cov(X,Y)=1/2$.
Then, $P(|X-Y|>6)$ is
\begin{enumerate}

    \item less than 1/6
    \item equal to 1/2
    \item equal to 1/3
    \item greater than 1/2

\end{enumerate}
\solution
  \input{solutions/ma/2001/24.tex}

\item Let \(X_1,X_2\)... be a sequence of independent and identically distributed random variable with
\begin{align}
    \pr{X_1=-1}=\pr{X_1=1}=1/2
\end{align}
Suppose for the standard normal random variable Z,
\begin{align}
&\pr{-0.1\leq Z \leq 0.1}=0.08 .\label{qneq} \\ \nonumber
&\text{If } S_n= \sum_{i=1}^{n^2} X_i ,\text{then} \lim_{n\to\infty} \pr{S_n >\frac{n}{10}} =
\end{align}
\begin{enumerate}
    \item 0.42
    \item 0.46
    \item 0.50
    \item 0.54
\end{enumerate}
\solution
  \input{solutions/ma/2007/55.tex}

\item  Consider an amusement park where visitors are arriving according to a Poisson process with rate 1. Upon arrival, a visitor spends a random amount of time in the park and then departs. The time spent by the visitors is independent of one another, as well as of the arrival process and have common probability density function 
\begin{align}
    f(x) = 
    \begin{cases}
        e^{-x}, & x > 0\\
        0,      & otherwise
    \end{cases}
\end{align}
If at a given point, there are 10 visitors in the park, and p is the probability that there will be exactly two more arrivals before the next departure, then $\frac{1}{p}$ equals.....
\solution
  
    Let the cumulative distribution function of the random variable X be given by 
    $$F_{X}(x)=\left\{
    \begin{array}{ll}
      0 & x<0 \\
      x & 0\leq x<1/2\\
      (1+x)/2 & 1/2\leq x <1\\
      1 & x\geq 1
    \end{array} 
    \right. $$
    Then $\pr{X=1/2}=?$

Given,
$$F_{X}(x)=\left\{
    \begin{array}{ll}
      0 & x<0 \\
      x & 0\leq x<1/2\\
      (1+x)/2 & 1/2\leq x <1\\
      1 & x\geq 1
    \end{array} 
    \right. $$
\begin{align}
\tag{24.1}
\pr{X=1/2}=\pr{X\leq 1/2}-\pr{X<1/2}
\end{align}
\begin{align}
\tag{24.2}
\implies \pr{X=1/2}=F_{X}\brak{\frac{1}{2}}-F_{X}\brak{\frac{1}{2}^-}\\
\tag{24.3}
\implies \pr{X=1/2}=\brak{1+1/2}/2-1/2
\end{align}
\begin{align}
\tag{24.4}
\therefore \pr{X=1/2}=1/4
\end{align}
\newpage
\begin{figure}[t!]
\centering
\includegraphics[width=8cm]{cdf_plot.png}
\caption{cdf of Random variable X}
\end{figure}



\item  Let $\{X_n\}_{n\ge 1}$ be a sequence of independent and identically distributed random variables each having uniform distribution on [0,3]. Let $Y$ be a random variable, independent of $\{X_n\}_{n\ge 1}$, having probability mass function
\begin{align}
\pr{Y=k} = 
\begin{cases}
\frac{1}{(e-1)k!} & k=1,2,3\cdots \\
0 & otherwise
\end{cases}
\end{align}
Then $\pr{max\{X_1,X_2,\cdots X_Y\}\le 1}$ equals ............\\
%
\solution
  \input{solutions/st/2021/48.tex}

\item The characteristic function of a random variable X is given by
\begin{align}
\phi_{X}\brak{t}
=
\begin{cases}
\frac{\sin{t}\cos{t}}{t}           & t \neq 0 \\
1        & t = 0
\end{cases}
\end{align}
Then $ P\brak{|X|\leq \frac{3}{2}} =$ 
%
\solution
  \input{solutions/st/2020/16/Assignment5.tex}
% %
% \item Let $\{X_j\}$ be a sequence of independent Bernoulli random variables with $\mathbb{P}(X_j=1) = \frac{1}{4}$ and let $Y_n = \frac{1}{n} \sum_{j=1}^{n}X_j^2$. Then $Y_n$ converges, in probability, to $\rule{2cm}{0.15mm}$ .
% %
% \solution
%   \input{solutions/ma/2018/25.tex}

%
\item Let $\{X_j\}$ be a sequence of independent Bernoulli random variables with $\mathbb{P}(X_j=1) = \frac{1}{4}$ and let $Y_n = \frac{1}{n} \sum_{j=1}^{n}X_j^2$. Then $Y_n$ converges, in probability, to $\rule{2cm}{0.15mm}$ .
\solution
  \input{solutions/ma/2018/25.tex}
%
\item The variable $x$ takes a value between $0$ and $10$ with uniform probability distribution. The variable $y$ takes a value between $0$ and $20$ with uniform probability distribution.The probability that sum of variables $(x+y)$ being greater than $20$ is
%
\item Robot Ltd. wishes to maintain enough safety
stock during the lead time period between
starting a new production run and its completion
such that the probability of satisfying the
customer demand during the lead time period
is 95\%. The lead time periods is 5 days and
daily customer demand can be assumed to follow
the Gaussian (normal) distribution with mean
50 units and a standard deviation of 10 units.
Using $\phi^{-1}$(0.95) =1.64 , where $\phi$ represents the
cumulative distribution function of the standard
normal random variable, the amount of safety
stock that must be maintained by Robot Ltd. to
achieve this demand fulfillment probability for
the lead time period is \rule{1cm}{0.15mm}  units (round off to
two decimal places).
%
\solution
  \input{solutions/me/2021/20.tex}
%
% \item 
% Probability density function $p(x)$ of random variable x is as shown below. The value of a is
% \begin{enumerate}[label=\Alph*)]
%     \item $\frac{2}{c}$
%     \item $\frac{1}{c}$
%     \item $\frac{2}{(b+c)}$
%     \item $\frac{1}{(b+c)}$
% \end{enumerate}
% \begin{figure}[h!]
% \centering
% \includegraphics[width=\columnwidth]{solutions/in/2006/2/figures/convolution.png}
% \caption{PDF}
% \label{in2006-2:fig:BSC}
% \end{figure}
% %
% \solution
%   \input{solutions/in/2006/2/latex3.tex}
% %
\item Let X be a random variable having probability density function 
\begin{align}
\label{st2021-22:eq:qpdf_X}
f\brak{x} = 
\begin{cases}
\frac{3}{13}(1-x)(9-x) & 0 < x < 1
\\
0 & \text{ otherwise}
\end{cases}
\end{align}
Then $\dfrac{4}{3} E [X(X^2 -15X + 27 ) ] $ equals --- ( round of to two decimal places). \\
\solution
  \input{solutions/st/2021/22.tex}
%
\item Two independent events E and F are such that $P(E\cap F) = \displaystyle\frac{1}{6}$,$P(E^c\cap F^c)=\displaystyle\frac{1}{3}$ and $P(E)>P(F)$. Then $P(E)$ is
\begin{enumerate}[label=(\Alph*)]
    \item $\displaystyle\frac{1}{2}$\\
    \item $\displaystyle\frac{2}{3}$\\
    \item $\displaystyle\frac{1}{3}$\\
    \item $\displaystyle\frac{1}{4}$
\end{enumerate}
%
\solution
  \input{solutions/ma/1999/1.28.tex}
%
\item Let $Y_{1},Y_{2},...,Y_{15}$ be a random sample of size 15 from the probability density function 
\begin{align}
\tag{Eq:1}
    f_{y}(y)=3(1-y)^{2} , 0<y<1
\end{align}
Use the central limit theorem to approximate $P\brak{\frac{1}{8}<\Bar{Y}<\frac{3}{8}}$
%
\solution
  \input{solutions/ma/1996/25.tex}
%
\item Let X and Y be two independent Poisson random variables with parameters 1 and 2 respectively. Then, $\pr{X=1 | X+Y = 4}$ is 
\begin{enumerate}[label=\Alph*)]
    \item 0.426
    \item 0.293
    \item 0.395
    \item 0.512
\end{enumerate}
%
\solution
  \input{solutions/ma/2006/16.tex}
%
\item If A and B are two events and the 
probability $\pr{B}\neq$1,then  
$\dfrac{\pr{A}-\pr{A B}}{1-\pr{B}}$ equals \\
\begin{enumerate}
    \item \pr{A|\bar{B}}\\
    \item \pr{A|B}\\
    \item \pr{\bar{A}|B}\\
    \item \pr{\bar{A}|\bar{B}}\\
\end{enumerate}   
%
\solution
  \input{solutions/ma/1997/1.18.tex}
%
\item Let E,F and G be mutually independent events with $P\brak{E}=\frac{1}{2}$,$P\brak{F}=\frac{1}{3}$ and $P\brak{G}=\frac{1}{4}$.Let p be the probability that at least two of the events among E,F and G occur.Then $12\times p=$
%
\solution
  \input{solutions/st/2020/19.tex}
%
\item Let (X,Y) be the coordinates of a point chosen at random inside the disc $x^2 + y^2 \leq r^2$ where $r\geq 0$. The probability that $Y \geq mX$ is
\begin{enumerate}[label = (\alph*)]
\setlength\itemsep{2em}
    \item $\dfrac{1}{2^r}$
    \item $\dfrac{1}{2^m}$
    \item $\dfrac{1}{2}$
    \item $\dfrac{1}{2^{r+m}}$
\end{enumerate}
%
\solution
  
    Let the cumulative distribution function of the random variable X be given by 
    $$F_{X}(x)=\left\{
    \begin{array}{ll}
      0 & x<0 \\
      x & 0\leq x<1/2\\
      (1+x)/2 & 1/2\leq x <1\\
      1 & x\geq 1
    \end{array} 
    \right. $$
    Then $\pr{X=1/2}=?$

Given,
$$F_{X}(x)=\left\{
    \begin{array}{ll}
      0 & x<0 \\
      x & 0\leq x<1/2\\
      (1+x)/2 & 1/2\leq x <1\\
      1 & x\geq 1
    \end{array} 
    \right. $$
\begin{align}
\tag{24.1}
\pr{X=1/2}=\pr{X\leq 1/2}-\pr{X<1/2}
\end{align}
\begin{align}
\tag{24.2}
\implies \pr{X=1/2}=F_{X}\brak{\frac{1}{2}}-F_{X}\brak{\frac{1}{2}^-}\\
\tag{24.3}
\implies \pr{X=1/2}=\brak{1+1/2}/2-1/2
\end{align}
\begin{align}
\tag{24.4}
\therefore \pr{X=1/2}=1/4
\end{align}
\newpage
\begin{figure}[t!]
\centering
\includegraphics[width=8cm]{cdf_plot.png}
\caption{cdf of Random variable X}
\end{figure}


  %
\item Let X be a non-constant positive Random Variable such that $E(X) = 9$.\\
Then which of the following statements is True?
\begin{enumerate}
\item  $E\brak{\frac{1}{X+1}} > 0.1$ and $\pr{X \ge 10} \le 0.9$
\item   $E\brak{\frac{1}{X+1}} < 0.1$ and $\pr{X \ge 10} \le 0.9$
\item   $E\brak{\frac{1}{X+1}} > 0.1$ and $\pr{X \ge 10} > 0.9$
\item   $E\brak{\frac{1}{X+1}} < 0.1$ and $\pr{X \ge 10} > 0.9$
\end{enumerate}
%
\solution
  \input{solutions/st/2021/1.tex}

%
\item Let F, G and H be pair wise independent events such that $\pr{F}=\pr{G}=\pr{H}=\dfrac{1}{3}$ 
and $\pr{F \cap G \cap H}=\dfrac{1}{4}$ Then the probability that at least one event among F, G and H occurs is 
\begin{enumerate}

\setlength\itemsep{2em}
\item $\dfrac{11}{12}$
\item $\dfrac{7}{12}$
\item $\dfrac{5}{12}$
\item $\dfrac{3}{4}$

\end{enumerate}
%
\solution
  \input{solutions/ma/2009/16.tex}
  %
  \item Let $\{ X_n \}_{n \geq 1}$ be a sequence of independent and identically distributed random variables each having uniform distribution on (0,2). For $n \geq 1$, let 
$Z_n = -\log_e \left( \prod\limits_{i=1}^n (2-X_i) \right)^\frac{1}{n}.$
Then, as $n \to \infty$, the sequence $\{ Z_n \}_{n \geq 1}$ converges almost surely to \rule{1cm}{0.15mm} (Round of to 2 decimal places).
%
\solution
  \input{solutions/st/2021/19/Assignment4.tex}
%
\item Let $X_1$, $X_2$ and $X_3$ be independent and identically distributed random variables with $E(X_1) = 0$ and $E\left(X^2_1\right)=\frac{15}{4}$. If $\psi : (0,\infty) \rightarrow (0,\infty)$ is defined through the conditional expectiation
$\psi(t) = E\left(X^2_1 | X_1^2 + X_2^2 + X_3^2 = t\right), t>0$
Then, $E(\psi((X_1+X_2)^2))$ is equal to,
%
\solution
  \input{solutions/ma/2015/28.tex}
%

\end{enumerate}
\end{document}
